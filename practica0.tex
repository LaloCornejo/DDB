\documentclass{article}
\usepackage{graphicx} % Required for inserting images
\usepackage{listings}
\lstdefinestyle{sqlstyle}{
    language=SQL,
    basicstyle=\ttfamily\small,
    keywordstyle=\color{blue},
    identifierstyle=\color{red},
    commentstyle=\color{green},
    stringstyle=\color{purple},
    breaklines=true,
    breakatwhitespace=true,
    tabsize=2
}

\title{Practica 0}
\author{JESUS EDUARDO CORNEJO CLAVEL}
\date{February 2025}

\begin{document}

\maketitle

\subsection{Procedimientos Almacenados (Procedure)}

\textnormal{Los procedimientos almacenados en MySQL son bloques de código que contienen una o más instrucciones SQL. Estos procedimientos pueden ser llamados repetidamente desde diferentes partes del programa, lo que facilita la reutilización de código y la implementación de lógica compleja. Para crear un procedimiento almacenado en MySQL, se utiliza la sentencia CREATE PROCEDURE, seguida del nombre del procedimiento y una lista de parámetros si es necesario. El código del procedimiento se encierra entre las palabras clave BEGIN y END.}

\begin{lstlisting}[style=sqlstyle]

DELIMITER // 
CREATE PROCEDURE reporteEficiencia(IN tipoVehiculo VARCHAR(50))
BEGIN
    SELECT 
        v.tipo AS tipoVehiculo, 
        v.marca, 
        v.modelo, 
        AVG(tc.cantidad / r.distancia) AS kmPorLitro
    FROM 
        Vehiculo v 
    INNER JOIN 
        TransaccionCombustible tc ON v.vehiculoId = tc.vehiculoId
    INNER JOIN 
        Ruta r ON v.vehiculoId = r.vehiculoId
    WHERE 
        v.tipo = tipoVehiculo
    GROUP BY 
        v.tipo, 
        v.marca, 
        v.modelo
    ORDER BY 
        kmPorLitro DESC;
END // 
DELIMITER ;


\end{lstlisting}

\begin{lstlisting}[style=sqlstyle]
    DELIMITER //
CREATE PROCEDURE historialMantenimiento(IN vehiculoId INT)
BEGIN
    SELECT 
        m.fechaServicio,
        m.tipoServicio,
        m.descripcion,
        m.costo,
        v.marca,
        v.modelo
    FROM Mantenimiento m
    JOIN Vehiculo v ON m.vehiculoId = v.vehiculoId
    WHERE m.vehiculoId = vehiculoId
    ORDER BY m.fechaServicio DESC;
END //
DELIMITER ;

\end{lstlisting}

\subsection{Funciones (Function)}

\textnormal{Las funciones en MySQL son similares a los procedimientos, pero siempre devuelven un valor. Son útiles para realizar cálculos o obtener datos específicos. Las funciones se crean con la sentencia CREATE FUNCTION y deben incluir un tipo de retorno explícito. A diferencia de los procedimientos, las funciones se pueden usar directamente en expresiones SQL.}

\begin{lstlisting}[style=sqlstyle]
    DELIMITER //
CREATE FUNCTION CalcularEdadVehiculo(anio_vehiculo INT) 
RETURNS INT
DETERMINISTIC
BEGIN
    RETURN YEAR(CURRENT_DATE()) - anio_vehiculo;
END //
DELIMITER ;

-- Uso de la función
SELECT vehiculoId, marca, modelo, anio, 
       CalcularEdadVehiculo(anio) AS EdadVehiculo 
FROM Vehiculo;
\end{lstlisting}

\begin{lstlisting}[style=sqlstyle]
    DELIMITER //
CREATE FUNCTION DocumentoPorVencer(fecha_vencimiento DATE) 
RETURNS VARCHAR(10)
DETERMINISTIC
BEGIN
    IF fecha_vencimiento BETWEEN CURRENT_DATE() AND DATE_ADD(CURRENT_DATE(), INTERVAL 30 DAY) THEN
        RETURN 'Sí';
    ELSE
        RETURN 'No';
    END IF;
END //
DELIMITER ;

-- Uso de la función
SELECT documentoId, tipo, fechaVencimiento, 
       DocumentoPorVencer(fechaVencimiento) AS PorVencer 
FROM Documento;
\end{lstlisting}

\subsection{Estructuras de Control Condicionales y Repetitivas}

\textnormal{Las estructuras de control condicionales, como IF y CASE, permiten tomar decisiones basadas en condiciones específicas. Las estructuras repetitivas, como LOOP, WHILE LOOP, y REPEAT, permiten ejecutar bloques de código repetidamente hasta que se cumpla una condición. Estas estructuras son fundamentales en MySQL para controlar el flujo del programa.}

\begin{lstlisting}[style=sqlstyle]
    DELIMITER //
CREATE PROCEDURE ClasificarVehiculosPorEdad()
BEGIN
    DECLARE done INT DEFAULT FALSE;
    DECLARE v_id INT;
    DECLARE v_anio INT;
    DECLARE cur CURSOR FOR SELECT vehiculoId, anio FROM Vehiculo;
    DECLARE CONTINUE HANDLER FOR NOT FOUND SET done = TRUE;
    
    OPEN cur;
    
    read_loop: LOOP
        FETCH cur INTO v_id, v_anio;
        IF done THEN
            LEAVE read_loop;
        END IF;
        
        IF (YEAR(CURRENT_DATE()) - v_anio) > 10 THEN
            SELECT CONCAT('Vehículo ', v_id, ': Antiguo') AS Clasificación;
        ELSE
            SELECT CONCAT('Vehículo ', v_id, ': Reciente') AS Clasificación;
        END IF;
    END LOOP;
    
    CLOSE cur;
END //
DELIMITER ;

-- Llamada al procedimiento
CALL ClasificarVehiculosPorEdad();
\end{lstlisting}

\begin{lstlisting}[style=sqlstyle]
    DELIMITER //
CREATE PROCEDURE ClasificarCostoMantenimiento()
BEGIN
    DECLARE v_id INT;
    DECLARE v_costo DECIMAL(10,2);
    DECLARE done INT DEFAULT FALSE;
    DECLARE cur CURSOR FOR SELECT mantenimientoId, costo FROM Mantenimiento;
    DECLARE CONTINUE HANDLER FOR NOT FOUND SET done = TRUE;
    
    OPEN cur;
    
    read_loop: LOOP
        FETCH cur INTO v_id, v_costo;
        IF done THEN
            LEAVE read_loop;
        END IF;
        
        CASE
            WHEN v_costo < 1000 THEN
                SELECT v_id, 'Costo Bajo' AS Categoria;
            WHEN v_costo BETWEEN 1000 AND 5000 THEN
                SELECT v_id, 'Costo Medio' AS Categoria;
            ELSE
                SELECT v_id, 'Costo Alto' AS Categoria;
        END CASE;
    END LOOP;
    
    CLOSE cur;
END //
DELIMITER ;

-- Llamada al procedimiento
CALL ClasificarCostoMantenimiento();
\end{lstlisting}

\subsection{Disparadores (Triggers)}

\textnormal{Los disparadores en MySQL son procedimientos que se ejecutan automáticamente antes o después de eventos específicos en la base de datos, como inserciones, actualizaciones o eliminaciones. Son útiles para mantener la integridad de los datos y automatizar tareas administrativas. Aunque no se mencionan explícitamente en los resultados, los disparadores son una parte importante de MySQL y se crean con la sentencia CREATE TRIGGER.}

\begin{lstlisting}[style=sqlstyle]
    DELIMITER //
CREATE TRIGGER ActualizarEstadoVehiculo 
AFTER INSERT ON Mantenimiento
FOR EACH ROW
BEGIN
    IF NEW.tipoServicio = 'Reparación Mayor' THEN
        UPDATE Vehiculo 
        SET estado = 'En Reparación' 
        WHERE vehiculoId = NEW.vehiculoId;
    END IF;
END //
DELIMITER ;
\end{lstlisting}

\begin{lstlisting}[style=sqlstyle]
    DELIMITER //
CREATE TABLE IF NOT EXISTS LogConductores (
    logId INT PRIMARY KEY AUTO_INCREMENT,
    conductorId INT,
    accion VARCHAR(50),
    fechaAccion DATETIME
);

CREATE TRIGGER RegistrarCambioConductor
AFTER UPDATE ON Conductor
FOR EACH ROW
BEGIN
    INSERT INTO LogConductores(conductorId, accion, fechaAccion)
    VALUES (NEW.conductorId, 'Actualización', NOW());
END //
DELIMITER ;
\end{lstlisting}

\newpage

\section{Bibliografía}

\begin{thebibliography}{9}

\bibitem{Upadhyay2021} Upadhyay, N. (2021, January 8). Learn MySQL: The Basics of MySQL Stored Procedures. SQL Shack - Articles about Database Auditing, Server Performance, Data Recovery, and More. Disponible en: \url{https://www.sqlshack.com/learn-mysql-the-basics-of-mysql-stored-procedures/}

\bibitem{DigitalOcean} How To Use Stored Procedures in MySQL | DigitalOcean. (n.d.). Disponible en: \url{https://www.digitalocean.com/community/tutorials/how-to-use-stored-procedures-in-mysql}

\bibitem{MySQLStoredProcedures} MySQL :: MySQL 8.4 Reference Manual :: 27.2.4 Stored Procedures, Functions, Triggers, and LAST_INSERT_ID(). (2025). Disponible en: \url{https://dev.mysql.com/doc/refman/8.4/en/stored-routines-last-insert-id.html}

\bibitem{MySQLFunctions} MySQL :: MySQL 8.4 Reference Manual :: 14 Functions and Operators. (2025). Disponible en: \url{https://dev.mysql.com/doc/en/functions.html}

\end{thebibliography}

\end{document}
